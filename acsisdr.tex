%% This is file `elsarticle-template-2-harv.tex',
%%
%% Copyright 2009 Elsevier Ltd
%%
%% This file is part of the 'Elsarticle Bundle'.
%% ---------------------------------------------
%%
%% It may be distributed under the conditions of the LaTeX Project Public
%% License, either version 1.2 of this license or (at your option) any
%% later version.  The latest version of this license is in
%%    http://www.latex-project.org/lppl.txt
%% and version 1.2 or later is part of all distributions of LaTeX
%% version 1999/12/01 or later.
%%
%% The list of all files belonging to the 'Elsarticle Bundle' is
%% given in the file `manifest.txt'.
%%
%% Template article for Elsevier's document class `elsarticle'
%% with harvard style bibliographic references
%%
%% $Id: elsarticle-template-2-harv.tex 155 2009-10-08 05:35:05Z rishi $
%% $URL: http://lenova.river-valley.com/svn/elsbst/trunk/elsarticle-template-2-harv.tex $
%%

%%\documentclass[preprint,authoryear,12pt]{elsarticle}

%% Use the option review to obtain double line spacing
%% \documentclass[authoryear,preprint,review,12pt]{elsarticle}

%% Use the options 1p,twocolumn; 3p; 3p,twocolumn; 5p; or 5p,twocolumn
%% for a journal layout:

%% Astronomy & Computing uses 5p
%% \documentclass[final,authoryear,5p,times]{elsarticle}
\documentclass[final,authoryear,5p,times,twocolumn]{elsarticle}

%% if you use PostScript figures in your article
%% use the graphics package for simple commands
%% \usepackage{graphics}
%% or use the graphicx package for more complicated commands
\usepackage{graphicx}
%% or use the epsfig package if you prefer to use the old commands
%% \usepackage{epsfig}

%% The amssymb package provides various useful mathematical symbols
\usepackage{amssymb}
%% The amsthm package provides extended theorem environments
%% \usepackage{amsthm}

\usepackage[pdftex,pdfpagemode={UseOutlines},bookmarks,bookmarksopen,colorlinks,linkcolor={blue},citecolor={green},urlcolor={red}]{hyperref}
\usepackage{hypernat}

%% The lineno packages adds line numbers. Start line numbering with
%% \begin{linenumbers}, end it with \end{linenumbers}. Or switch it on
%% for the whole article with \linenumbers after \end{frontmatter}.
%% \usepackage{lineno}

%% natbib.sty is loaded by default. However, natbib options can be
%% provided with \biboptions{...} command. Following options are
%% valid:

%%   round  -  round parentheses are used (default)
%%   square -  square brackets are used   [option]
%%   curly  -  curly braces are used      {option}
%%   angle  -  angle brackets are used    <option>
%%   semicolon  -  multiple citations separated by semi-colon (default)
%%   colon  - same as semicolon, an earlier confusion
%%   comma  -  separated by comma
%%   authoryear - selects author-year citations (default)
%%   numbers-  selects numerical citations
%%   super  -  numerical citations as superscripts
%%   sort   -  sorts multiple citations according to order in ref. list
%%   sort&compress   -  like sort, but also compresses numerical citations
%%   compress - compresses without sorting
%%   longnamesfirst  -  makes first citation full author list
%%
%% \biboptions{longnamesfirst,comma}

% \biboptions{}

\journal{Astronomy \& Computing}

\begin{document}

\begin{frontmatter}

%% Title, authors and addresses

%% use the tnoteref command within \title for footnotes;
%% use the tnotetext command for the associated footnote;
%% use the fnref command within \author or \address for footnotes;
%% use the fntext command for the associated footnote;
%% use the corref command within \author for corresponding author footnotes;
%% use the cortext command for the associated footnote;
%% use the ead command for the email address,
%% and the form \ead[url] for the home page:
%%
%% \title{Title\tnoteref{label1}}
%% \tnotetext[label1]{}
%% \author{Name\corref{cor1}\fnref{label2}}
%% \ead{email address}
%% \ead[url]{home page}
%% \fntext[label2]{}
%% \cortext[cor1]{}
%% \address{Address\fnref{label3}}
%% \fntext[label3]{}

\title{Automated reduction of submillimetre single-dish heterodyne
  data from the James Clerk Maxwell Telescope using ORAC-DR}

%% use optional labels to link authors explicitly to addresses:
%% \author[label1,label2]{<author name>}
%% \address[label1]{<address>}
%% \address[label2]{<address>}

\author[jac]{Malcolm J.\ Currie\corref{cor1}}
\ead{m.currie@jach.hawaii.edu}
\author[jac,cornell]{Tim Jenness}
\ead{tjenness@cornell.edu}
\author[jac]{Remo P.\ J.\ Tilanus\fnref{rpt}}
\author[jac]{Brad Cavanagh}
\author[jac]{David S. Berry}
\author[jac]{Jamie Leech\fnref{jxl}}
\author[jac]{Luca Rizzi\fnref{lr}}

\cortext[cor1]{Corresponding author}
\fntext[rpt]{Present address: Leiden Observatory, PO Box 9513, 2300 RA
  Leiden, The Netherlands}
\fntext[jxl]{Present address: Department of Physics, University of
  Oxford, Denys Wilkinson Building, Keble Road, Oxford, OX1 3RH, UK}
\fntext[lr]{Present address: W.\ M.\ Keck Observatory, 65-1120 Mamalahoa Hwy, Kamuela,
  HI 96743, USA}

\address[jac]{Joint Astronomy Centre, 660 N.\ A`oh\=ok\=u Place, Hilo, HI
  96720, USA}
\address[cornell]{Department of Astronomy, Cornell University, Ithaca,
  NY 14853, USA}

\begin{abstract}
%% Text of abstract

With the advent of modern multi-receptor heterodyne instruments that
can result in observations generating thousands of spectra per minute it is
no longer feasible to reduce these data as individual spectra. This
paper describes how baselined, mosaicked data cubes can be created by
an automated pipeline in a reliable manner, including detection of bad
spectra, removal of transient effects and creation of high fidelity
moment maps.

\end{abstract}

\begin{keyword}
%% keywords here, in the form: keyword \sep keyword

%% MSC codes here, in the form: \MSC code \sep code
%% or \MSC[2008] code \sep code (2000 is the default)

submillimeter astronomy \sep
methods: data analysis \sep
pipelines \sep
techniques: spectroscopic \sep
techniques: image processing \sep
James Clerk Maxwell Telescope

\end{keyword}

\end{frontmatter}

% \linenumbers

%% Journal abbreviations
\newcommand{\mnras}{Mon Not R Astron Soc}
\newcommand{\aap}{Astron Astrophys}
\newcommand{\pasp}{Pub Astron Soc Pacific}
\newcommand{\apj}{Astrophys J}
\newcommand{\qjras}{Quart J R Astron Soc}
\newcommand{\an}{Astron.\ Nach.}
\newcommand{\ijimw}{Int.\ J.\ Infrared \& Millimeter Waves}

%% main text
\section{Introduction}
\label{sec:intro}

As heterodyne receivers have progressed from single-receptor
instruments
\citep{1992IJIMW..13.1487P,1992IJIMW..13..647D,1992IJIMW..13.1827C} to
small focal plane arrays
\citep{2003SPIE.4855..322G,2004A&A...423.1171S} to 16 element arrays
such as HARP at JCMT \citep{2009MNRAS.399.1026B}, and beyond
\citep{2007stt..conf..264G}, and correlators have improved such that
we can easily obtain spectra at 10\,Hz with 8192 channels, data rates
have increased substantially such that it is now common-place to take
a short observation resulting in thousands of spectra. This is only
going to become worse with the advent of instruments with 64,000
channels and large format arrays such as CHAI on CCAT
\citep{2009ASPC..417..113R}.

In submillimeter astronomy data reductions packages such as
\htmladdnormallinkfoot{CLASS}{http://www.iram.fr/IRAMFR/GILDAS} and
SPECX \citep{SPECX} were developed that worked well with
single-receptor instruments. Scripting interfaces and tools for
curating collections of spectra were insufficient as the data rates
increased and data pipelines were suggested
\citep[e.g.,][]{1995ASPC...75..117W}. The ACSIS data reduction system
\citep{2000ASPC..216..502L,2000SPIE.4015..114H}, delivered to the JCMT
in 2005, aimed to deal with the data rate issues by providing a
pipeline that sent the calibrated spectra, with optional baselining,
to a component that would place the spectra in to a data
cube. Although it was possible to store the raw data in CASA
measurement sets \citep{2012ASPC..461..849P}\footnote{At the time this
  was being developed CASA was known as AIPS++}, the data cube was the product that was archived and
taken away by the astronomer for further analysis. This strategy was
forced on us given the computer resources available when ACSIS was
being designed and developed and was known to have risks associated
with it. Coadding spectra into the cube allowed for impressive ``data
compression'' for stare and jiggle observing modes although the gains
were much less in scanning observing modes. There were many downsides
though. The observing system required that the cube parameters be
specified and initially it was felt that the observer should select
the parameters in the Observing Tool \citep{2002ASPC..281..453F}. It
was also necessary that the observer specify the baseline regions and
any frequency binning required. The gridding and the frequency binning
were irreversible and the observer needed to ensure they did not make
a mistake. Additionally, baselining with anything other than a DC
offset was also problematic as the fit parameters for every spectrum
were not stored and so could not be reversed. The burden placed on the
observer having to specify everything, whilst simultaneously not
making a mistake, was not acceptable and by 2006 we realised that
computers were fast enough and storage large enough to be able to
write the calibrated spectra directly to disk and defer further data
reduction to a pipeline.

\section{Heterodyne Data Reduction Pipeline}

It was decided that data reduction recipes would be written for the
existing ORAC-DR pipeline infrastructure in use at JCMT
\citep{TFAJenness2011,2008AN....329..295C} with key requirements that
the output data cube should be specified from the telescope pointing
information and the location of each receptor on the sky, and that
baseline regions should be determined automatically by examining the
spectra as an ensemble. Furthermore, as progress was made on the
pipeline we additionally realised that we should also be able to
detect bad spectra and remove them from the coadd as well as doing
quality assurance tests. The latter were critical for the JCMT Legacy
Survey projects
\citep{2007PASP..119..855W,2009ApJ...693.1736W,2007PASP..119..102P}
who wanted to ensure they received data of consistent quality.

The JCMT heterodyne pipeline has two operating modes. The default behaviour is
for the pipeline to generate the best possible data products without
regard to efficiency. This is generally what is required by scientists
at their institutions and the mode we run in at the JCMT Science
Archive \citep{2008ASPC..394..565J}. The other mode is a cut-down
version of the recipes that runs at the JCMT itself during
observing. This pipeline has constrained timing requirements and can
not perform many of the advanced processing features provided by the
main pipeline. It's role is to provide simple quality assurance
information and basic coadds to the observer and it will not be
discussed further in this paper.

\subsection{Determining Cube Parameters}

Description from David on how MAKECUBE works out the correct
gridding. Include how we handle moving sources.

\subsection{Automated Baseline Removal}

The real meat: smoothing, basic baseline subtraction, group coadd,
clump finding, unmakecube, source masking, iteration.

\subsection{Determination of moment maps}

Clump finding and baseline masking critical.

\subsection{Removal of Bad-Baseline Spectra}

Malcolm's ADASS XXII poster.

\subsection{Flatfielding}

Early HARP data from 2006 seemed to have problems with the relative
calibration of the receptors. A self flatfielding algorithm was
developed that relied on the receptors on average seeing the same
signal for large scan maps of molecular clouds
\citep{2010MNRAS.401..455C} and this had some success in removing
striping from integrated intensity images.

Now discuss the pipeline implementation. Before and after flatfielding examples.

\subsection{Recipe Tuning}

Some mention of the ability to bin up the frequency axis.

\subsection{Alternative Recipes}

Something about the other recipes such as line forest and broad
line. Don't need much detail.

\section{Processing of Legacy Data}

Mention GSD, modifications for multi-subband support. Sub-band merging
in general. How we can use the same pipeline processing to make many
additional years of JCMT data available in a reduced form with
incremental effort.

Would be nice if we could reduce some GSD data and compare it with a
published map from SPECX. Something like Horsehead but I can't find
that actually published anywhere.


\section{Conclusion}

Works great.

\section{Acknowledgments}

The James Clerk Maxwell Telescope is operated by the Joint Astronomy
Centre on behalf of the Science and Technology Facilities Council of
the United Kingdom, the Netherlands Organisation for Scientific
Research, and the National Research Council of Canada. We thank the
many JCMT support scientists and survey scientists who have tested the
pipeline. In particular we thank Jessica Dempsey, Holly Thomas, Jan
Wouterloot and Jennifer Hatchell.

This work was built on the Starlink Software Collection which was
developed by the Starlink Project until 2005
\citep{1982QJRAS..23..485D,2005ASPC..347...22D} and then opened up to
the community. The source code for the Starlink software and ORAC-DR
is open-source and is available at
\htmladdnormallink{https://github.com/Starlink}.

%% The Appendices part is started with the command \appendix;
%% appendix sections are then done as normal sections
%% \appendix

%% \section{}
%% \label{}

%% References
%%
%% Following citation commands can be used in the body text:
%%
%%  \citet{key}  ==>>  Jones et al. (1990)
%%  \citep{key}  ==>>  (Jones et al., 1990)
%%
%% Multiple citations as normal:
%% \citep{key1,key2}         ==>> (Jones et al., 1990; Smith, 1989)
%%                            or  (Jones et al., 1990, 1991)
%%                            or  (Jones et al., 1990a,b)
%% \cite{key} is the equivalent of \citet{key} in author-year mode
%%
%% Full author lists may be forced with \citet* or \citep*, e.g.
%%   \citep*{key}            ==>> (Jones, Baker, and Williams, 1990)
%%
%% Optional notes as:
%%   \citep[chap. 2]{key}    ==>> (Jones et al., 1990, chap. 2)
%%   \citep[e.g.,][]{key}    ==>> (e.g., Jones et al., 1990)
%%   \citep[see][pg. 34]{key}==>> (see Jones et al., 1990, pg. 34)
%%  (Note: in standard LaTeX, only one note is allowed, after the ref.
%%   Here, one note is like the standard, two make pre- and post-notes.)
%%
%%   \citealt{key}          ==>> Jones et al. 1990
%%   \citealt*{key}         ==>> Jones, Baker, and Williams 1990
%%   \citealp{key}          ==>> Jones et al., 1990
%%   \citealp*{key}         ==>> Jones, Baker, and Williams, 1990
%%
%% Additional citation possibilities
%%   \citeauthor{key}       ==>> Jones et al.
%%   \citeauthor*{key}      ==>> Jones, Baker, and Williams
%%   \citeyear{key}         ==>> 1990
%%   \citeyearpar{key}      ==>> (1990)
%%   \citetext{priv. comm.} ==>> (priv. comm.)
%%   \citenum{key}          ==>> 11 [non-superscripted]
%% Note: full author lists depends on whether the bib style supports them;
%%       if not, the abbreviated list is printed even when full requested.
%%
%% For names like della Robbia at the start of a sentence, use
%%   \Citet{dRob98}         ==>> Della Robbia (1998)
%%   \Citep{dRob98}         ==>> (Della Robbia, 1998)
%%   \Citeauthor{dRob98}    ==>> Della Robbia


%% References with bibTeX database:

\bibliographystyle{model2-names}
\bibliography{acsisdr.bib}

%% Authors are advised to submit their bibtex database files. They are
%% requested to list a bibtex style file in the manuscript if they do
%% not want to use model2-names.bst.

%% References without bibTeX database:

% \begin{thebibliography}{00}

%% \bibitem must have one of the following forms:
%%   \bibitem[Jones et al.(1990)]{key}...
%%   \bibitem[Jones et al.(1990)Jones, Baker, and Williams]{key}...
%%   \bibitem[Jones et al., 1990]{key}...
%%   \bibitem[\protect\citeauthoryear{Jones, Baker, and Williams}{Jones
%%       et al.}{1990}]{key}...
%%   \bibitem[\protect\citeauthoryear{Jones et al.}{1990}]{key}...
%%   \bibitem[\protect\astroncite{Jones et al.}{1990}]{key}...
%%   \bibitem[\protect\citename{Jones et al., }1990]{key}...
%%   \harvarditem[Jones et al.]{Jones, Baker, and Williams}{1990}{key}...
%%

% \bibitem[ ()]{}

% \end{thebibliography}

\end{document}

%%
%% End of file `elsarticle-template-2-harv.tex'.
